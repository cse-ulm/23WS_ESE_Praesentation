\documentclass[10pt,a4paper]{beamer}

%\usepackage[applemac]{inputenc}
\usepackage[ngerman]{babel}
\usepackage[utf8]{inputenc}
\usepackage{times}
\usepackage{graphicx}
\usepackage{setspace} %Zeilenabstand?
\usepackage{wrapfig} %Benoetigt fuer Textumflossene Bilder
\usepackage{hyperref}
\usepackage{marvosym} %Symbol Link
%%%%%%%%%%%%%%%%%%%%%%%%%%%%%%%%%%%%%%%%%%%%%%%%%%%%%%%%%%%%%%%%%%%%%%%%%%%%%%%%
\usetheme{Boadilla}
\definecolor{ecs100}{RGB}{29, 45, 68}
\setbeamercolor{structure}{fg=ecs100,bg=white}

\setbeamertemplate{section in toc}[default]

\setbeamertemplate{itemize items}[circle]
\setbeamertemplate{enumerate items}[default]



\setbeamertemplate{headline}
{%
\vspace*{2.5ex}%
\begin{beamercolorbox}[wd=0.09\textwidth,ht=2ex,dp=0.5ex,leftskip=.5em,rightskip=.5em]{author in head/foot}%
\usebeamerfont{author in head/foot}%
Folie \insertframenumber%
\end{beamercolorbox}%
\vspace*{+0.02ex}%
\hspace*{0.07\textwidth}%
\begin{beamercolorbox}[wd=0.92\textwidth,ht=1.9ex,dp=0.5ex,right,leftskip=.5em]{title}%
\begin{picture}(0,0)
\put(0,0.5){\rule[1.8ex]{\textwidth}{0.2ex}}\end{picture}% 2ex - 0.2ex = 1.8ex
{\usebeamerfont{title in head/foot}%
\qquad \insertshorttitle \ \ $\vert$ \ \ \insertshortsubtitle \ \ $\vert$ \ \ \insertshortdate \hfill \insertsubsection $\leftarrow$\insertsection \break
}
\end{beamercolorbox}%
}

\setbeamertemplate{frametitle}
{%
\vspace*{1ex}%
\usebeamercolor{title} \bf \large \insertframetitle%
\vspace*{-0.5ex} }

\setbeamertemplate{footline}{} % Fußzeile aktivieren/deaktivieren
\setbeamertemplate{navigation symbols}{} % Navigationsfunktion aktivieren/deaktivieren
%%%%%%%%%%%%%%%%%%%%%%%%%%%%%%%%%%%%%%%%%%%%%%%%%%%%%%%%%%%%%%%%%%%%%%%%%%%%%%%%%%%%%%%%%%%
%Die eigenen Daten hier einfügen:

\date[WS 2022/2023]{Ulm, WS 2022/2023}
\title[CSE]{Studienbeginn CSE - WS 22/23}
\subtitle[Studienbeginn]{\hspace{1cm} Fachschaft CSE}

\setcounter{framenumber}{-1}

%MY DEFINITIONS BEGIN
\def\r#1{{\color{red}#1}}
\def\g#1{{\color{green}#1}}
\def\mg#1{{\color{ecs100}#1}}
\def\titlefont#1{\textbf{\large\mg{#1}}}
%MY DEFINTIONS END

%BEGIN DOCUMENT
\begin{document}


%Titel Seite
\frame[plain]{
	\vspace*{-0.3cm}
	\flushright \includegraphics[width=.2\textwidth]{logo_fs-cse.png}
%	\leftskip-2.58em%
	\vspace{-0.2cm}
    \begin{center}
		\makebox[\textwidth]{\includegraphics[width=\paperwidth]{tischkicker-wide.jpg}}
    \end{center}

	%\vspace*{1cm}
	\begin{center}
		\textcolor{ecs100}{\hspace{-5cm}\Large  \inserttitle}
	\end{center}
}


% Das Inhaltsverzeichnis


%%%%%%%%%%%%%%%%%%%%%%%%%%%%%%%%%%%%%%%%%%%%%%%%%%%%%%%%%%%%%%%%%%%%%%%%%%%%%%%%%%%%%%%%%%%
%Ab hier beginnt die Präsentation:
% 1. Folie: Inhaltsverzeichnis
\begin{frame}
	\frametitle{Inhaltsverzeichnis}
	\tableofcontents
	% [pausesections] % Animation/Einblendung der einzelnen Abschnitte
\end{frame}

%%%%%%%%%%%%%%%%%%%%%%%%%%%%%%%%%%%%%%%%%%%%%%%%%%%%%%%%%%%%%%%%%%%%%%%%%%%%%%%%%%%%%%%%%%%
%Ab hier beginnt die Präsentation:

\section{eduroam}
\begin{frame}
\frametitle{eduroam}
%\makebox[\textwidth]{\includegraphics[width=\paperwidth]{eduroam.png}}
\begin{center}
\includegraphics[width=0.5\paperwidth]{eduroam.png}
\end{center}
%\includegraphics[width=0.5\paperwidth]{eduroam.png}
\begin{itemize}
		\item https://www.uni-ulm.de/einrichtungen/kiz/service-katalog/netzwerk-konnektivitaet/wlan/eduroam/
\end{itemize}
% https://www.uni-ulm.de/einrichtungen/kiz/service-katalog/netzwerk-konnektivitaet/vpn/
\end{frame}

\section{Grundsätzliches}
\begin{frame}
\frametitle{Grundsätzliches}
\begin{center}
\includegraphics[width=0.8\paperwidth]{logos.png}
\end{center}
\vspace{0.3cm}
	\begin{itemize}
		\item Kooperationsstudiengang der Universität und Technischen Hochschule Ulm
		\begin{itemize}
			\setlength{\itemsep}{10pt} %Zeilenabstand ändern
			\item Zwei Studierendenausweise
			\item Doppeltes Angebot beim Hochschulsport, Orchestern etc.
			\item Unterschiedlichen Regelungen
			\item Vorlesungen an verschiedensten Standorten
		\end{itemize}
	\end{itemize}
\end{frame}

\begin{frame}
\begin{center}
		\makebox[\textwidth]{\includegraphics[width=\paperwidth]{Ulm.png}}
    \end{center}
\end{frame}

\begin{frame}
\begin{center}
		\makebox[\textwidth]{\includegraphics[width=\paperwidth]{karte.png}}
    \end{center}
\end{frame}

\section{Vorlesungszeiten}
\begin{frame}
\frametitle{Vorlesungszeiten}
	\begin{itemize}
	\setlength{\itemsep}{10pt} %Zeilenabstand ändern
		\item Semesterbeginn / -ende
		
		\begin{itemize}
		\setlength{\itemsep}{10pt} %Zeilenabstand ändern
			\item THU: 27.09.2022 - 20.01.2023 \\ (Vorlesungsfreie Zeit um Weihnachten: 24.12.2022 - 09.01.2023)
			\item Uni: 17.10.2022 - 18.02.2023\\(Vorlesungsfreie Zeit um Weihnachten: 24.12.2022 - 05.01.2023)
		\end{itemize}
	
		\item cum tempore / sine tempore:
		\begin{itemize}
		\setlength{\itemsep}{10pt} %Zeilenabstand ändern
			\item Uni: c.t., d.h. die Vorlesung beginnen \emph{Viertel nach}
			\item HS: s.t., d.h. die Vorlesungen beginnen \emph{um Punkt}
			\item Bei Prüfungen: Hinweise beachten
		\end{itemize}
	\end{itemize}
\end{frame}

\section{Studentenausweise}
\begin{frame}
\frametitle{Studentenausweise}
\begin{itemize}
	\setlength{\itemsep}{10pt} %Zeilenabstand ändern
	\item Zwei Studentenausweise mit identischer Matrikelnummer
	\item Verschiedene Konten auf den einzelnen Ausweisen
	\item Verlängerung jedes Semester jeweils an einem Automaten der THU und einem SB-Terminal in der Uni (PIN wird benötigt!)
	\item Kostenlose Fahrt im DING-Netz ab 18h sowie am Wochenende und an Feiertagen	
	\item Beim Kauf eines Semestertickets unterschiedliche Semesterzeiträume der THU und Universität beachten
\end{itemize}
\end{frame}

\subsection*{Studentenausweis Universität}
\begin{frame}
\frametitle{Studentenausweis Universität}
\begin{itemize}
	\setlength{\itemsep}{10pt} %Zeilenabstand ändern
	\item Zugangskarte zu PC-Pools sowie nachts zur Universität
	\item Bezahlmittel mit drei verschiedenen Konten:
	\begin{itemize}
		\setlength{\itemsep}{10pt} %Zeilenabstand ändern
		\item \textbf{Mensa} (Aufwertung durch Terminals oder am Info-Point) / \\ Bezahlen auch in den Mensen der THU möglich
		\item \textbf{Parken} (Aufwertung durch Automaten bei den Parkplätzen)
		\item \textbf{Druckkontingent} (Umbuchung am SB-Terminal, 16 Euro pro Kalenderjahr frei)
		\item \textbf{Druck- und Kopierkonto} (für Drucker von Ricoh, zusätzlich)
	\end{itemize}
	\item Identifikationsmittel bei Prüfungen
\end{itemize}
\end{frame}

\subsection*{Studentenausweis Technische Hochschule}
\begin{frame}
\frametitle{Studentenausweis Technische Hochschule}
\begin{itemize}
	\setlength{\itemsep}{10pt} %Zeilenabstand ändern
	\item Bezahlmittel an den Mensen der Technische Hochschule oder Universität
	\item Identifikationsmittel beim Drucken % (10 Euro pro Semester frei)
	\item Zugang zu den \textbf{kostenlosen} Parkplätzen der Technische Hochschule
	\item Identifikation bei Prüfungen
\end{itemize}
\end{frame}

% \section{Modulwahl}
% \begin{frame}
% \frametitle{Modulwahl}
% \begin{itemize}
% 	\setlength{\itemsep}{10pt} %Zeilenabstand ändern
% 	\item \href{https://www.uni-ulm.de/mawi/mawi-cse/studierende/master-cse/wahlpflicht-ma-cse/}{\Mundus~Allgemeine Informationen}
% 	\item \href{https://goo.gl/R8cisD}{\Mundus~Modulhandbuch Uni}
% 	\item \href{https://goo.gl/Ad3BQm}{\Mundus~Modulhandbuch HS}
% \end{itemize}
% \end{frame}

\section{Accounts und Portale}
\begin{frame}
\frametitle{Accounts und Portale}
\begin{itemize}
	\setlength{\itemsep}{10pt} %Zeilenabstand ändern
	\item E-Mails
		\begin{itemize}
		\setlength{\itemsep}{10pt} %Zeilenabstand ändern
			\item https://sogo.uni-ulm.de/SOGo/
			\item https://webmail.hs-ulm.de
		\end{itemize}
	\item LSF
		\begin{itemize}
		\setlength{\itemsep}{10pt} %Zeilenabstand ändern
			\item \textbf{https://campusonline.uni-ulm.de}
			\item https://lsf.verwaltung.hs-ulm.de
		\end{itemize}
	\item Moodle
		\begin{itemize}
		\setlength{\itemsep}{10pt} %Zeilenabstand ändern
			\item https://moodle.uni-ulm.de
			\item https://moodle-thu.de
		\end{itemize}
\end{itemize}
\end{frame}

\section{Prüfungsanmeldung}
\begin{frame}
\frametitle{Prüfungsanmeldung}
\begin{itemize}
	\setlength{\itemsep}{10pt} %Zeilenabstand ändern
	\item Anmeldung spätestens \textbf{vier Tage vor der Prüfung}, danach ist auch ein Rücktritt von der Anmeldung nicht mehr möglich
	\item Alle Prüfungen müssen über das Portal der Uni Ulm angemeldet werden, \textbf{nicht} über die Hochschule
	\item bei Fragen an die \href{https://www.uni-ulm.de/mawi/mawi-cse/studienfachberatung/}{\Mundus~Studienfachberatung CSE} wenden
	\begin{itemize}
		\setlength{\itemsep}{10pt} %Zeilenabstand ändern
		\item Beate Mayer (Universität Ulm) - beate.mayer@uni-ulm.de
		\item Kirsten Huss (Hochschule Ulm) - huss@hs-ulm.de
	\end{itemize}
\end{itemize}
\begin{alertblock}{Sonderregelung der Universität im WS}
Die Anmeldung muss weiterhin spätestens vier Tage vor der Prüfung geschehen, aber dann ist noch bis ein Tag vor der Prüfung ein Abmeldung möglich!
\end{alertblock}
\end{frame}
\section{Fachschaft CSE}
\begin{frame}
\frametitle{Fachschaft CSE}
\begin{itemize}
	\setlength{\itemsep}{10pt} %Zeilenabstand ändern
	\item FS CSE ist nicht eigenständig, sondern ein Teil der FS Mathe
	\item Treffen: Alle zwei Wochen im Vorlesungszeitraum, genaue Informationen auf unserer Website.
	\item Cloud mit Altklausuren / Material aus älteren Semestern
	\item Erreichbar über:
	\begin{itemize}
		\item \href{https://www.facebook.com/groups/356850697732243/}{\Mundus~Facebook}
		\item \href{https://chat.whatsapp.com/CQYUbnH2UGZLXuFFrI0VxO}{\Mundus~WhatsApp}
		\item E-Mail: fs-cse@uni-ulm.de
	\end{itemize}
\end{itemize}	
\end{frame}

\begin{frame}
\frametitle{ \href{http://fs-cse.de}{\Mundus~fs-cse.de}}
\begin{center}
\includegraphics[width=0.85\textwidth]{website.png}
\end{center}
\end{frame}
\begin{frame}
\frametitle{\href{http://cloud.fs-cse.de}{\Mundus~cloud.fs-cse.de} }
\begin{itemize}
		\setlength{\itemsep}{10pt} %Zeilenabstand ändern
		\item Sammlung von Skripten, Übungsaufgaben, Altklausuren
		\item Registrierung über \href{http://cloud.fs-cse.de/register/}{\Mundus~cloud.fs-cse.de/register/} (\textbf{nur} im Uninetz, oder via VPN/webVPN)
		\item Jeder kann in den Ordner \emph{Datenaustausch} Dateien hochladen
		\item Diese Präsentation liegt im Ordner CSE-Bachelor
	\end{itemize}
\end{frame}

\begin{frame}
	\frametitle{UniNow}
	\begin{center}
	\includegraphics[height=0.7\textheight]{uninow1.png}
	\hspace{1cm}
	\includegraphics[height=0.7\textheight]{uninow2.png}
	\end{center}
	\begin{block}{Folge uns auf UniNow}
	https://uninow.page.link/FS-CSE
	\end{block}
\end{frame}

\begin{frame}
	\begin{center}
		\Huge{Noch Fragen?}
	\end{center}
\end{frame}

\end{document}